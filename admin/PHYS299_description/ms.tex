% this is a LaTeX lol file
%----------------------------------------------------------------------
% name:
%   hogg_cv.tex
% purpose:
%   Hogg's CV and list of publications
% to-do:
%   - Convert all \ads{} to \doi{} -- the DOIs should exist now.
%   - Is the ``what constrains'' paper published anywhere?
%   - Should I make an \acronym{} macro that uses \small?  I should.
% comments:
%   - Don't list full author lists of 10 or more.
%   - If you want to include/remove the date, see \renewcommand{\today}.
%   - Swap 1/0 flag to include invited talks.
%----------------------------------------------------------------------

\documentclass[11pt,letterpaper]{article}
\usepackage{cleanTeX}
\usepackage{pagecolor}
\usepackage{multicol}

\usepackage{natbib}
\bibliographystyle{apj}

\usepackage{xstring}

% put in the date -- or not
%  \renewcommand{\today}{2018 October 20}

% define colors

\usepackage{lipsum}
\usepackage{amsmath}
\usepackage{bm}

\newcommand{\beq}{\begin{equation}}
\newcommand{\eeq}{\end{equation}}

\fancypagestyle{firstpage}{%
  % \lhead{\deemph{\texttt{cleanTeX v0.1}}}
}

\begin{document}
\thispagestyle{firstpage}%\sloppy\sloppypar\raggedbottom%\frenchspacing
\ifdefined\dark
    \pagecolor{bggrey}
    \color{white}
\fi

\def\parskipdefault{\parskip}
\setlength{\parskip}{1.2ex}

\def\parindentdefault{\parindent}
\setlength{\parindent}{0pt}

\textbf{The Damping Wing in $z\sim5-6$ Quasars}

Angus Beane and Adam Lidz

\texttt{abeane@sas.upenn.edu}

\deemph{\today}

\section{Background}
The Epoch of Reionization (EoR) is the final phase transition of the universe
as the diffuse, neutral intergalactic medium (IGM) is ionized by the first
generation of galaxies. Observations of this era are limited. The main
constraints on the timing of the EoR come from measurements of the optical
depth from the cosmic microwave background (CMB). Recent measurements place
$\tau \sim 0.05$, which corresponds to a reionization midpoint of $z_r \sim
7.5$, depending on the exact ionization history used
\citep{2018arXiv180706209P}. Since the form of the ionization history is not
known, this allows for various reionization scenarios, but most plausible
scenarios place the end of reionization to be prior to $z\sim6$.

In a related problem, the Ly$\alpha$ optical depth of individual quasars is
known to show a large degree of spatial fluctuation, signficantly larger than
expected from density fluctuations alone \citep[e.g.][]{2015PASA...32...45B}.
Typical explanations for this problem either make use of a fluctuating UV
background \citep{2017MNRAS.465.3429C} or

Our project will attempt to either measure or place an upper limit on the bulk
neutral fraction at $z<6$. Since the optical depth for the Ly$\alpha$
transmission is high, a neutral fraction of only $\sim10^{-4}$ is necessary to
completely block transmission. Therefore, it is impossible to naively identify
a region with no Ly$\alpha$ transmission as being caused by a neutral hydrogen
cloud or a region of slight residual netural fraction.

We will attempt 

\bibliography{references}

\end{document}
