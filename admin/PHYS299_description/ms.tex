\documentclass[11pt,letterpaper]{article}
\usepackage{cleanTeX}
\usepackage{pagecolor}
\usepackage{multicol}

\usepackage{natbib}
\bibliographystyle{apj}

\usepackage{xstring}

% put in the date -- or not
%  \renewcommand{\today}{2018 October 20}

% define colors

\usepackage{lipsum}
\usepackage{amsmath}
\usepackage{amssymb}
\usepackage{bm}

\newcommand{\beq}{\begin{equation}}
\newcommand{\eeq}{\end{equation}}

\fancypagestyle{firstpage}{%
  % \lhead{\deemph{\texttt{cleanTeX v0.1}}}
  \lhead{\deemph{\texttt{}}}
}

\begin{document}
\thispagestyle{firstpage}%\sloppy\sloppypar\raggedbottom%\frenchspacing
\ifdefined\dark
    \pagecolor{bggrey}
    \color{white}
\fi

\def\parskipdefault{\parskip}
\setlength{\parskip}{1.2ex}

\def\parindentdefault{\parindent}
\setlength{\parindent}{0pt}

\textbf{The Damping Wing in $z\sim5-6$ Quasars}

Angus Beane and Adam Lidz

% \texttt{abeane@sas.upenn.edu}

\deemph{\today}

\section{Background}
The Epoch of Reionization (EoR) is the final phase transition of the universe
as the diffuse, neutral intergalactic medium (IGM) is ionized by the first
generation of galaxies. Observations of this era are limited. The main
constraints on the timing of the EoR come from measurements of the optical
depth from the cosmic microwave background (CMB). Recent measurements place
$\tau \sim 0.05$, which corresponds to a reionization midpoint of $z_r \sim
7.5$, depending on the exact ionization history used
\citep{2018arXiv180706209P}. Since the form of the ionization history is not
known, this allows for various reionization scenarios, but most plausible
scenarios place the end of reionization to be prior to $z\sim6$.

In a related problem, the Ly$\alpha$ optical depth of individual quasars is
known to show a large degree of spatial fluctuation, signficantly larger than
expected from density fluctuations alone \citep[e.g.][]{2015PASA...32...45B}.
Typical explanations for this problem either make use of a fluctuating UV
background \citep{2017MNRAS.465.3429C} or line-of-sight variation in
temperature \citep{2015ApJ...813L..38D}.

Our project will attempt to either measure or place an upper limit on the bulk
neutral fraction at $z<6$. Since the optical depth for the Ly$\alpha$
transmission is high, a neutral fraction of only $\sim10^{-4}$ is necessary to
completely block transmission. Therefore, it is impossible to naively identify
a region with no Ly$\alpha$ transmission as being caused by a neutral hydrogen
cloud or a region of slight residual netural fraction.

\section{Description of Project}
We will attempt to determine if significant neutral regions remain at $z<6$,
which would show that the EoR is not complete by $z=6$, by studying the dark
gaps in quasar spectra. We will be relying upon a technique developed by
\citet{2015ApJ...799..179M} which uses the damping wing to distinguish between
regions of order unity and $\sim10^{-4}$ neutral fraction. Since significant
absorption far from line center (i.e. $\Delta v \sim 300 \text{km}/\text{s}$)
occurs only when the neutral fraction of the cloud is close to unity, one can
detect a ``wing'' of transmission instead of a sharp recovery to mean flux
\citep[Figure~4][]{2015ApJ...799..179M}. Such a feature cannot be detected in
a single spectra --- however, by stacking several spectra together, we hope to
be able to measure the damping wing feature.

We plan to primarily use the set of $z\sim5.7-6.5$ spectra from
\citet{2018ApJ...864...53E}. If necessary, we will also use older spectra from
\citet{2006AJ....132..117F}.

\bibliography{references}

\end{document}
